\documentclass{article}
\usepackage[margin=1in]{geometry}
\usepackage{enumerate}

\begin{document}

% Don't change this portion-- it will format the section below for you. :)
\newcommand{\coverpage}{
  \onecolumn
  \begin{center}
    \LARGE\labtitle \\ \bigskip \bigskip \large\name
    \\ \bigskip \labdate
    \end{center}
  \setcounter{page}{1} }

%% Change these variables
\newcommand{\labtitle}{CS118: Project 1 Report}
\newcommand{\name}{Leslie Lam, 804-302-387 \\ Dominic Mortel, 904-287-174 \\
Kevin Huynh, 704-283-105}
\newcommand{\labdate}{May 1, 2016}

%% This creates your lab cover page.
\coverpage

\section{HTTPMessage Implemenation}
    The HttpMessage class is an abstract class that is extended by the
    HttpRequest and HttpResponse classes. These classes are used to handle the
    encoding and decoding of HTTP messages. The classes are defined in http-
    message.h and http-message.cpp. The HttpMessage class contains several
    functions; these are a few of the more important ones:
    \begin{itemize}
        \item decodeFirstLine: Abstract function that decodes first line of
            HTTP Message.
        \item encode: Abstract function to encode the HTTP Message.
        \item encodeHeaders: Encode the HTTP Message headers as a vector of
            chars.
        \item decodeHeaderLine: Given a vector of bytes, decode the header line
            and store in map.
    \end{itemize}
    The HttpRequest class contains the following important functions:
    \begin{itemize}
        \item decodeFirstLine: Decodes first line of HTTP Request.
        \item encode: Encodes the HTTP Request as a vector of chars (bytes).
    \end{itemize}
    The HttpResponse class contains the following important functions:
    \begin{itemize}
        \item decodeFirstLine: Decodes first line of HTTP Response.
        \item encode: Encodes the HTTP Response as a vector of chars (bytes).
    \end{itemize}
    
    The biggest difference between the HttpRequest and HttpResponse classes is
    their implementation of the function decodeFirstLine, due to the
    differences in the first line of each type of HTTP message. The most
    important function in the HttpMessage class is the decodeHeaderLine
    function, which takes a string or stream of bytes and extracts the header
    and its value. These are then placed in a map that maps headers to values.
    
    These classes were used to implement HTTP messages in our web server and
    web client. The server and client build the HTTP message from scratch with
    from a stream of bytes with decode. The messages are sebt as a stream of
    bytes after calling the encode function.

\section{Web Server}
    The web server is a basic web server that supports HTTP 1.0 with timeouts
    and works as follows: \\ \\
    \textbf{Listening for Connections:}
        The server creates a socket that listens to the hostname and port
        specified in the arguments. If no hostname and port are specified, then
        the server listens to localhost at port 4000. The server then serves
        files from the directory specified in the argument list, otherwise it
        serves from the current directory. \\ \\
    \textbf{Accepting Connections:}
        When the web server receives a new connection, it creates a child
        process with the fork() system call. The child process then handles
        serving the requested file. Because of these parallel processes, the
        web server has the capability to handle concurrent connections. The
        server supports three different status codes: 200 OK, 400 Bad Request,
        404 Not Found, 408 Request Timeout. If the server does not receive data
        from the client connection for five seconds and the whole HTTP request
        has not been transmitted yet, then the server will time out and send a
        response with the 408 Request Timeout status. \\ \\
    \textbf{Parsing Request:}
        The first line of the request is parsed separately to get the requested
        file information. The remaining headers are decoded line by line; the
        method is the same as that used in the client, which is described
        below. \\ \\
    \textbf{Sending Response:}
        After parsing every line of the request, the server then attempts to
        send the requested file. The URL of the request is parsed to find the
        first singular slash. Everything after the first singular slash
        indicates the file requested. The server attempts to open the file, and
        sends a 404 Not Found response if the file could not be opened. After
        opening the file, the server sends a 200 OK response and then the file
        if possible. After the response has been sent, the connection is
        closed.

\section{Web Client}
    The client side works as follows: \\
    \textbf{Parse URL:}
        The client first parses the URL argument to obtain the host name, port
        number, and the requested file name. Using these parameters, we create
        a TCP socket to connect to the server. The HTTP GET request message is
        always the same no matter the server, with only the URL and host name
        being changed depending on the use. \\ \\
    \textbf{Read Response from Server:}
        The client reads the response from the socket using the ``recv"
        function, storing the values in a 4096 byte buffer. We decode the
        response data into two sections: 1) the headers and 2) the payload
        which are then stored in an HttpResponse object. These two sections
        were parsed using the following:
        \begin{enumerate}
            \item Each header is separated by a CRLF (\textbackslash
                r\textbackslash n) and we utilize this property to properly
                decode them.
            \begin{itemize}
                \item Initially we parse only the first header which contains
                    status information about the request by processing data
                    from the beginning of the buffer until the first CRLF.
                \item From there we move our iterator forward by two bytes so
                    that it is located at the beginning of the next header,
                    move the second iterator to the next CR and decode the
                    line. We repeat this process until we decode all headers,
                    using ``recv" each time we reach the end of the current
                    buffer data.
                \begin{itemize}
                    \item If the end of the buffer is reached before the header
                        portion of the message is completely received, then the
                        first iterator is set back to the beginning of the
                        buffer.

                    \item If the ``recv" function does not receive any information
                        for more than 5 seconds, it times out and closes the
                        socket.
                    
                    \item The client implements some logic so that if there are
                        two consecutive CRLF bytes, it knows that any more data
                        afterward will be a part of the payload.
                \end{itemize}
            \end{itemize}

            \item After obtaining the entire HTTP Response header fields, the client
                looks through the headers of the message:
                \begin{itemize}
                    \item If the status of the response is `404' or `400,' the client
                        is informed of the error, nothing is downloaded, the socket is
                        closed, and the program finishes.
                    \item If the status is a success, the client will use the file name
                        and create a file with that name using ofstream. A special case
                        is implemented to change the file name of `/' into
                        `index.html.'
                \end{itemize}
            \item We create a payload buffer of 256KB and read from the socket in 256KB
                chunks and write to the file.
                \begin{itemize}
                    \item We loop using ``recv" to read 256KB at a time and write
                        it to the file. If the ``recv" call returns 0, this means the
                        socket has been closed and the whole file has been sent.
                    
                    \item If the ``recv" call does not receive any data for five seconds
                        before the entire message has been received, the
                        client will timeout and close the socket.
                        
                \end{itemize}
        \end{enumerate}
        The socket is then closed and the program finishes.

\section{Problems Encountered}
    For the web server, the biggest issue that we encountered was handling edge
    cases when parsing the URL of the request. We had a lot of bugs (corrupted
    downloading, file not being created, etc.) stemming from an incorrectly
    parsed URL. Because the request may not necessarily have http:// at the
    beginning, we had decided to parse the URL by searching for the first
    singular slash and keeping everything after it. We ran into issues when the
    first singular slash was the last character or if the slash was not
    present. We first tried to use sscanf to find the first slash; however, we
    ran into bugs when the slash was at the end of the string. We eventually
    implemented it iteratively with a for loop and some if statements.

    Another problem we faced was with handling large files. Originally we thought
    that HTTP/1.0 required the Content-Length header, and we used that information
    to create the size of our payload buffer. However, for large files (such as 100MB
    or larger), this would create an enormous vector of chars. As a result, the
    kernel would kill our client for using too much memory. To solve this issue,
    we decided to read from the socket and write to the file in chunks of 256KB. We
    determined that the file was finished sending then the recv call returned 0,
    meaning that the socket was closed.

\section{Testing}
    When we built our web server, we tested it with a web browser first. We
    connected by forwarding a private IP address and port number from our
    vagrant instance and having our web server listen to that IP address and
    port number. First we tested serving basic files such as text files and
    html files. We then expanded our file types to include jpg, png, and pdf
    files. We defaulted our content type to octet-stream so that if a file type
    was not recognized, it could still be downloaded by the client. We then
    tested our web server with our web client to ensure that the server could
    handle interesting URL cases, such as if the client requests a directory
    instead of a file.
    
    We tested our client by creating different types of files (.html, .txt,
    .jpg, etc..)  in our server's directory and downloading from our server.
    Some of the things we tested were attempting to download a directory,
    different types of files, files within subdirectories, and a `index.html'
    file using `/' as the URL. We did these tests because the client should be
    able to download many types of files, including files in subdirectories. We
    did not want any corner case to be missed. This testing was important
    because we discovered a lot of bugs with both our client and server by
    attempting to download other types of files.

    Next, we tested our client by downloading from other servers. We attempted
    to download images off of random websites and the CS 118 Project Specs to
    test our client. These tests were done to make sure that our client could
    work independently of the server and that the file would be downloaded
    correctly even if it was a very large size.

\section{Contributions}
    Leslie Lam (804-302-387) wrote the web server. Kevin Huynh (704-283-105)
    and Dominic Mortel (904-287-174) wrote the web client. Each wrote the
    corresponding sections of the report.

\end{document}